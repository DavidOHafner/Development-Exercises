\documentclass[12pt]{article}

\title{Exercise 31}
\author{Samantha Hafner}
\date{10/9/19}

\begin{document}
\maketitle

$M_2$ pushes a $\$$ and $i$ $a$'s when run on input string $a^i\{a,b,c\}^{(n-i)}$ of length $n$. $i$ can be as great as but not greater than $n$, meaning the maximum stack depth on all inputs of length $n$ is $n+1$ and, therefore, the space complexity of $M_2$ is $\Theta(n)$.

The space complexity of a pushdown automaton is always less than or equal to the time complexity of that pushdown automaton because no transition can increase the stack length by more than 1. Therefore, the time complexity of $M_2$ is $\Omega(n)$.

Each transition in $M_2$ which is part of a cycle on the state graph and may therefore be performed more than once either consumes a character from the input or a character from the stack. Once a character is removed from the stack, no more characters are added to it, so the number of transitions that do not consume characters from the input is at most a constant plus the maximum stack size which is $O(n)$. The number of transitions which do consume a character from the input is $O(n)$. The total number of transitions is the sum of these two, which is still $O(n)$.

Because the time complexity of $M_2$ is $\Omega(n)$ and $O(n)$, the time complexity of $M_2$ is $\Theta(n)$.

\end{document}
