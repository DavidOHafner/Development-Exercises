\documentclass[12pt]{article}

\title{Exercise 47}
\author{Samantha Hafner}
\date{11/17/19}

\usepackage{amsfonts}
\usepackage{amsmath}

\begin{document}
\maketitle

I conjecture the following classifications, but provide full proof for only (a) and (e):

(a) i. Proof follows.

(b) iii for typical encodings. Easily decidable. Can be proved non context-free for some encodings via the pumping lemma.

(c) iii. Easily decidable. For pumping length $p$, the string $\textrm{a}^p\textrm{b}^{2p}\textrm{a}^p$ cannot be partitioned into $uvxyz$ and pumped due to the requirement that $|vxy|\le p$.

(d) iv for typical encodings. Recognized by running in parallel $M$ on all possible inputs, increasing allowable run time and input length simultaneously. If it were decidable, then $E_{TM}$ would be decidable by checking if every state is accessible for every subset of states, and accepting iff the accept state is not in the largest accepted subset.

(e) ii Proof follows.


\bigskip

The set $L_1 = \{u\in\{0,1,2\}^n\;|\;7\textrm{ does not divide }\sum_{i=0}^nu_i3^{n-i}\}$ is recognized by deterministic finite automata $M_1 = (\mathbb{Z}_7,\; \mathbb{Z}_3,\; \delta(\sigma, x) \equiv 3\sigma+x \textrm{ mod } 7,\; 0,\\\mathbb{Z}_7\setminus\{0\})$ and is, therefore, regular. At each point in computation $M_1$'s state is equivalent to the numeral processed that far mod 7. $M_1$ accepts if and only if the whole numeral is not divisible by 7. The existence of $M_1$ demonstrates that $L_1$ is regular.

\bigskip

Let $L_2$ be the set such that $u\in L_2$ if and only if $u=\emptyset$ or there exist $s,t\in L_2$ such that $u=[s]t$. (adapted from \textit{Mathematics for Computer Science}, by Eric Lehman, F. Thomson Leighton, and Albert R. Meyer, revision of June 6, 2018, page 222.) Consider the push down automata:
$$M_2 = (\{q_0,q_1,q_2\},\;  \{[,]\},\;  \{\#,[\},\;  \delta(x)=\begin{cases}
\{(q_1,\#)\}                & x=(q_0,\epsilon,\epsilon)\\
\{(q_1,[)\}                 & x=(q_1,[,\epsilon)\\
\{(q_1,\epsilon)\}          & x=(q_1,],[)\\
\{(q_2,\epsilon)\} & x=(q_1,\epsilon,\#)\\
\emptyset                   & \textrm{otherwise}
\end{cases},\;  q_0, \{q_2\})$$
$M_2$ accepts $\emptyset$. Assuming $M_2$ properly handles all inputs smaller than $u$, $M_2$ accepts $u$ if there exist $s,t\in L_2$ such that $u=[s]t$ by adding \# and [ to the stack, processing $s$, resulting in an unchanged stack, removing the [ from the stack, resulting in a stack consisting of a single \#, and processing $t$ on that stack, which accepts. Further, $M_2$ rejects all strings that start with a ], and only accepts a string $u$ that start with a [ by pushing [ to the stack, processing an accepting string $s$, resulting in an unchanged stack, processing a ], and then processing an accepting string $t$. This means that $M_2$ will only accept a string accept a string $u$ if there exist $s,t\in L_2$ such that $u=[s]t$. By structural induction, $M_2$ recognizes $L_2$. The existence of $M_2$ demonstrates that $L_2$ is context-free.

Any deterministic finite automaton which recognizes $L_2$ must be in a distinct state after processing each of \{[, [[, [[[, ...\} because each corresponds to a different set of possible accepting continuations. This necessitates an infinite number of states. $L_2$ is not regular.

$L_2$ is context-free but not regular.



\end{document}
